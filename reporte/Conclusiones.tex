%\chapter{Conclusiones}

\begin{thebibliography}{X}
	\bibitem{1} Tejada, Marcia (2013). Jarduino, Sistema de riego manejado por Arduino (Documento in�dito). Seminario: Introducci�n a la programaci�n de microcontroladores con tecnolog�as libres. UNQ.
	\bibitem{2} Gil-Albert V. F,(2006),Manual t�cnico de jardiner�a I. Establecimiento de jardines, parques y espacios verdes, Madrid Espa�a, Mundi-prensa.
	\bibitem{3} Torrecilla C.M,(1998),Manual practico de la jardiner�a, Madrid Espa�a,EL PA�S.
	\bibitem{4} Daniel Rios Cruellas. (2016). Sistema de riego autom�tico. 2016, de universitat polit�cnica de catalunya Sitio web: https://upcommons.upc.edu/handle/2117/105385
	\bibitem{5} Fernando Escamilla Mart�nez . (2016). AUTOMATIZACI�N Y TELECONTROL	DE SISTEMAS DE RIEGO. febrero,20,2016, de UNIVERSIDAD POLITECNICA DE VALENCIA.
	\bibitem{6} Gabriel Escalas Rodr�guez . (2014). Dise�o y desarrollo de un prototipo de riego autom�tico controlado con Raspberry Pi y Arduino Y TELECONTROL DE SISTEMAS DE RIEGO. febrero,6,2014, de Universidad politenica de catalunya Sitio web:https://upcommons.upc.edu/bitstream/handle/2099.1/25074/
	memoria.pdf?sequence=4\&isAllowed=y
	\bibitem{7} Armando Torrente Trujillo ,Iv�n Dar�o M�ndez G , Edward Iv�n L�pez R. . (2015). Dise~no de un sistema de riego por microaspersi�n automatizado para el cultivo de guan�bana mediante el uso de las herramientas SIG. Agosto,6,2014, de Universidad Surcolombiana Neiva
	
	\bibitem{8} Mart�n Tarjuelo. (2010). El riego y su tecnolog�a. M�xico: Mundi-Prensa
	\bibitem{9} Francisco Miguel �guila Mar�n. (2008). Agricultura T�cnica en M�xico. M�xico: Universidad
	Hohenheim.
	\bibitem{10} Yessica S�ez. (2017). Sistema de Riego Inteligente para Optimizar el Consumo de	Agua. Panam�: Universidad Tecnol�gica de Panam�.
	\bibitem{11}  Comisi�n estatal del agua (2007). Gu�a de ahorro y reutilizaci�n del agua. Guanajuato, p.12.
\end{thebibliography}