En (Tejada, 2013) encontramos un proyecto de riego autom�tico y aut�nomo para administrar riego por goteo en peque�os jardines y balcones. Es sistema utiliza un Arduino Uno que se encarga de obtener datos del ambiente y procesarlos para activar un mecanismo de distribuci�n de agua cuando se cumplan una serie de condiciones\cite{1}.\\

En (Gil-Albert, 2006) se describe de manera detallada las especies de pastos que existen, las caracter�sticas de los mismos y los cuidados que requieren. Tambi�n encontramos informaci�n y caracter�sticas de otras plantas de jardines\cite{2}.\\

En (Torrecilla, 1998) encontramos un sistema de riego inteligente mediante una aplicaci�n en donde se decriben los diferentes tipos de suelos para jardines, ademas de sus caracteristicas, componentes y consideraciones a tener en cuenta al momento de sembrar
y cuidar plantas\cite{4}.\\

En(Rios,2016) el documento recoge el dise�o del hardware y software de un programador de riego completo, econ�mico y f�cil de utilizar. A partir de un microcontrolador y tratando las se�ales provenientes de sensores anal�gicos de humedad, luminosidad y temperatura y de un detector de lluvia digital, se controlar�an cinco electrov�lvulas que ser�an activadas y desactivadas en un tiempo determinado y cuando se den las caracter�sticas que desee el usuario\cite{5}.\\

En(Escamilla,2016) la elaboraci�n de este tranajo sugiere de la necesidad de plantear soluciones, para optimizar el uso de agua para regado hortofruticolas, considerando que es tambi�n una herramienta util para el aprovechamiento de los recursos hidricos y con
ello aumentar los niveles de producci�n y de calidad obtenitendo mediante estas tecnicas.
Este trabajo tambien se introduce en el mundo de los microprocesadores, tratando conceptos de electronica, programaci�n y transmision de se�ales\cite{6}.\\

En (Escalas,2015) En este proyecto se propone automatizar los sistemas de riego que instala una empresa de jardiner�a. Para ello se va a crear una plataforma que va a permitir al usuario ver los datos meteorol�gicos y tener un control total sobre su sistema de riego.
Este sistema engloba una serie de sensores conectados a un micro-controlador, a su vez controlado por un micro-procesador con salida a internet, lo que permitir�a controlar la aplicaci�n a distancia. Tanto los sensores como los dos dispositivos son de bajo coste\cite{7}.\\

En (Tarjuelo, 2010) se presenta un proyecto el cual consiste en un sistema de riego en el que se implementan tecnolog�as con el prop�sito de conseguir una id�nea utilizaci�n del agua. Este sistema de riego tiene como objetivo suministrar a las plantas agua de forma eficiente y sin alterar la fertilidad del suelo, as� como interrelacionar los principales componentes de un sistema de riego los cuales son: energ�a, agua, mano de obra y sistematizaci�n\cite{8}.\\

En ( �guila, 2008) se presenta un sistema de riego automatizado en tiempo real con balance h�drico y medici�n de humedad del suelo. Este sistema se enfoca principalmente en el monitoreo por medio de tecnolog�a para determinar el momento oportuno y cantidad de riego de un cultivo o de una planta\cite{9}.\\

En(S�ez, 2017) nos habla sobre un sistema de riego inteligente el cual optimiza el consumo de agua de los cultivos y de jardines, la principal actividad de este sistema es proporcionar una estrategia de distribuci�n para minimizar el consumo total de agua suministrado a cada l�nea de riego bas�ndose en datos obtenidos de cada planta\cite{10}.\\




