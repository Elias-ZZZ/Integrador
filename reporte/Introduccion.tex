
\section{Definici�n del problema}
D�cada a d�cada la cantidad de agua potable en M�xico se ve disminuida debido al desperdicio inconsciente de la misma. Tan solo en Guanajuato, un habitante promedio consume al rededor de 87 litros al dia., siendo aproximadamente 50\% no reutilizable.
Entre las actividades donde se presenta un mayor consumo de agua se encuentra el riego de jardines dom�sticos por lo que, la implementaci�n de un sistema de riego automatizado ayuda a dar un uso m�s racional del agua, logrando en el proceso una mejora en la condici�n del jard�n.

\newpage
\section{Objetivos}
\subsection{Objetivo General}
Desarrollar e implementar un sistema de riego autom�tico que reduzca el consumo agua al momento de regar jardines dom�sticos, utilizando un circuito electr�nico que sea capaz de tomar decisiones basado en las condiciones ambientales del jard�n.
\subsection{Objetivos espec�ficos}
\begin{itemize}
	\item Elaborar un circuito electr�nico que implemente los sensores necesarios para monitorear	el ambiente.
	\item Dise�ar un circuito el�ctrico en el que se implemente los sensores necesarios para el monitoreo del jard�n.
\end{itemize}
